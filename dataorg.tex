\documentclass[aspectratio=169,12pt,t]{beamer}
\usepackage{graphicx}
\setbeameroption{hide notes}
\setbeamertemplate{note page}[plain]
\usepackage{listings}
\usepackage{eepic}

\input{header.tex}

%%%%%%%%%%%%%%%%%%%%%%%%%%%%%%%%%%%%%%%%%%%%%%%%%%%%%%%%%%%%%%%%%%%%%%
% end of header
%%%%%%%%%%%%%%%%%%%%%%%%%%%%%%%%%%%%%%%%%%%%%%%%%%%%%%%%%%%%%%%%%%%%%%

% title info
\title{data organization in spreadsheets}
\subtitle{}
\author{\href{https://kbroman.org}{Karl Broman}}
\institute{Biostatistics \& Medical Informatics, UW{\textendash}Madison}
\date{\href{https://kbroman.org}{\tt \scriptsize \color{foreground} kbroman.org}
\\[-4pt]
\href{https://github.com/kbroman}{\tt \scriptsize \color{foreground} github.com/kbroman}
\\[-4pt]
\href{https://twitter.com/kwbroman}{\tt \scriptsize \color{foreground} @kwbroman}
\\[-4pt]
{\scriptsize Slides: \href{https://kbroman.org/Talk_DataOrg}{\tt kbroman.org/Talk\_DataOrg}}
}


\begin{document}

% title slide
{
\setbeamertemplate{footline}{} % no page number here
\frame{
  \titlepage

  \vfill \hfill \includegraphics[height=6mm]{Figs/cc-zero.png} \vspace*{-5mm}

\note{
  These are slides for a talk for the OSGA Webinar Series
  on 24 Sept 2021, based on
  my paper of the same title with Kara Woo,
  {\tt doi.org/gdz6cm}.

  Slides: \href{https://kbroman.org/Talk_DataOrg/dataorg.pdf}{\tt
    kbroman.org/Talk\_DataOrg/dataorg.pdf} \\
  Slides with notes: \href{https://kbroman.org/Talk_DataOrg/dataorg_notes.pdf}{\tt
    kbroman.org/Talk\_DataOrg/dataorg\_notes.pdf} \\
  Source: \href{https://github.com/kbroman/Talk_DataOrg}{\tt
    github.com/kbroman/Talk\_DataOrg}


}

} }



\begin{frame}[c]{}


  \figh{Figs/spreadsheet_ugly.pdf}{0.8}


  \note{
    Spreadsheets are super useful for storing and organizing datasets.
    But how should you arrange the data in a spreadsheet?

    Often, data are arranged as they might appear in a table in a
    paper. This may be pleasing to view, but it can be make
    down-stream analysis difficult.

    Data analysts often spend a lot of time rearranging data prior to
    analysis. Here, I'll talk about how to organize data in
    spreadsheets, for ease of analysis.
  }

\end{frame}





\begin{frame}[c]{}


  \figh{Figs/dataorg_paper.png}{0.7}

  \vspace{10mm}
  \hfill \href{https://doi.org/gdz6cm}{\tt \lolit doi.org/gdz6cm}
  \vspace{-10mm}

  \note{
    A particular collaborator's data (not that shown on the previous
    slide) led me to write a website on data organization, which then
    led to this paper.
  }

\end{frame}





\begin{frame}{\emph{American Statistician}}


  \vspace{14mm}

  \figw{Figs/amstat_most_read.png}{1.0}

  \vspace{12mm}
  \hfill
  \href{https://bit.ly/amstat_most_read}{\tt \lolit bit.ly/amstat\_most\_read}

  \note{
    The ``Data organization in spreadsheets'' paper is the third-most
    downloaded paper at American Statistician, after two papers about
    p-values. This is by far my most widely-read article.
  }

\end{frame}





\begin{frame}[c]{}

  \centering
  \Large

  Be consistent

  \note{
    First, be consistent.

    Even if you make idiosyncratic choices, if you follow them in a
    consistent way, it will be easier to handle the product.
  }

\end{frame}


{
% invert colors on this slide
\setbeamercolor{normal text}{bg=foreground, fg=background}
\color{background}

\begin{frame}<handout:0>[c]{}

  \addtocounter{framenumber}{-1}

  \centering
  \fontsize{66pt}{66}\selectfont

  Be consistent

\end{frame}
}


\begin{frame}{Consistent categories}

  \bigskip

  \figh{Figs/consistent_cats.pdf}{0.7}

  \note{
    Be consistent in your labels on categories. (Ideally, use short
    but meaningful labels.)
  }

\end{frame}



\begin{frame}{Consistent missing values}

  \bigskip

  \figh{Figs/consistent_nas.pdf}{0.7}

  \bigskip

  \only<2>{\centerline{\large And no {\tt \vhilit 999} or {\tt \vhilit -999}!}}

  \note{
    Be consistent in your code for missing values. And don't use a
    code that is a valid number (like 999), as it might be missed in
    later analyses. (I mean, they might be included, but with that
    crazy value.) I prefer to not have empty cells, because it is less
    clear whether it is a mistake or intentional.
  }

\end{frame}



\begin{frame}{Consistent subject IDs}

  \bigskip

  \figh{Figs/consistent_ids.pdf}{0.7}

  \note{
    Be consistent in the format of subject IDs. OMG I spend so much
    time converting subject IDs between formats.

    I prefer to not use raw numbers, but it's nice to have them be
    short.
  }

\end{frame}




\begin{frame}{Consistent column names}

\fighnc{Figs/spreadsheet_colnames1.pdf}{0.8}

\vspace*{-0.6\textheight}
\hspace*{0.1\textwidth}
\fighnc{Figs/spreadsheet_colnames2.pdf}{0.8}

  \note{
    If you have multiple batches of the same measurements, use the
    same column names in each file.
  }

\end{frame}



\begin{frame}{Consistent layout}


\fighnc{Figs/spreadsheet_colnames1.pdf}{0.8}

\vspace*{-0.6\textheight}
\hspace*{0.1\textwidth}
\fighnc{Figs/spreadsheet_colnames2.pdf}{0.8}

\vspace*{-0.6\textheight}
\hspace*{0.2\textwidth}
\fighnc{Figs/spreadsheet_colnames3.pdf}{0.8}

  \note{
    And keep those columns in the same order. Use the same layout for
    the multiple spreadsheets.
  }

\end{frame}




\begin{frame}[c]{Consistent date format}

  \figh{Figs/dates.pdf}{0.8}

  \note{
    Use a single format for all dates. I don't remember what the e's
    were for in this example, but sometimes using 4 digits for the
    year and sometimes 2 can be painful to deal with.
  }

\end{frame}



\begin{frame}[c]{}

\vspace{24pt}

\figh{Figs/iso_8601.png}{0.8}

\vfill

\hfill {\tt \footnotesize \lolit \href{http://xkcd.com/1179/}{xkcd.com/1179}}



  \note{
    And really, you should just go with the ISO 8601 format for
    writing dates, for ease of sorting, and because we should all use
    the same format.
  }

\end{frame}





\begin{frame}[c]{}

\figwboxed{Figs/oct4_tweet_1.png}{0.7}

\figwboxed{Figs/oct4_tweet_2.png}{0.7}

  \note{
    And dates in Excel are a abomination. It likes to turn non-dates
    into dates, and it stores dates internally as an integer, but with
    different starting values on different systems.

    My preference is to force Excel to treat certain columns as text.

    One might also enter dates as numbers `YYYYMMDD`, or split them
    into 3 columns (year, month, day).
  }

\end{frame}






\begin{frame}[fragile]{Consistent file names}

\vspace*{-24pt}

\begin{semiverbatim}
\begin{lstlisting}
Complete F2 Liver TG Set.xls
CPL Rosetta Lipids FINAL.xls
D2O Summary of All F2 Samples MF 30July2009.xls
FINAL RBM Data 102989 26Sept2007.xls
Mapped Urine Plasma Data to Statgen.xls
Necropsy Tracking Report rk 2011-04-26.xls
Necropsy Tracking Report rk61412.xls
Necropsy_Tracking_Report_rk_052912_atb.xls
Original Necropsy Tracking Report rk.xls
RBM Tube Number Key.xls
\end{lstlisting}
\end{semiverbatim}

\vspace*{-24pt}

\only<2>{
  \bi
\item No spaces or special characters
\item Short but descriptive
\item Consistent scheme
\item Take advantage of computer sorting
  \ei
}

  \note{
    Where you can, you should also strive for some consistent system
    for naming files. And for later analysis, it would be best to
    avoid spaces or other special characters (though underscores and
    hyphens are good).

    File names should be descriptive of their contents, so you don't
    need to look inside to understand what it contains.

    Take advantage of the way the computer sorts files, by starting
    with general groupings, followed by more specific groupings.
  }

\end{frame}



\begin{frame}[c]{No ``{\hilit final}'' in file names}

\vspace*{3mm}

\centering

% comic from http://www.phdcomics.com/comics/archive.php?comicid=1531
\figh{Figs/phd101212s.png}{0.8}

\note{
  Never include ``final'' in a file name. Best to use version numbers.
  No file is ever final.
}

\end{frame}




\begin{frame}{Choose good names for things}

  \bigskip

  \figw{Figs/variable_names_table.png}{1.0}


  \bigskip
  \bigskip
  \bigskip
  \hfill
\href{https://www.datacarpentry.org/spreadsheet-ecology-lesson/02-common-mistakes}{\lolit
  \tt DataCarpentry.org}


  \note{
    More generally, you want to put thought into the names that you
    choose for things, such as the names of columns. Don't include
    spaces, and be short but descriptive.
  }

\end{frame}




\begin{frame}[c]{No empty cells}

  \only<1>{\figh{Figs/skipping_cells.pdf}{0.8}}
  \only<2|handout 0>{\figh{Figs/skipping_cells_2.pdf}{0.8}}

  \note{
    Don't leave any cells empty, particularly as here where just the
    first of several repeated values are shown. If the rows of this
    spreadsheet get sorted, the information in the date column will be
    lost.

    In general, I prefer to have some missing value code inserted
    where data are missing, rather than leave cells blank. This helps
    to distinguish actually missing data from mistakes.
  }
\end{frame}




\begin{frame}{One thing per cell}

  \only<1>{\fighnc{Figs/one_thing_per_cell.pdf}{0.7}}

  \only<2|handout 0>{\fighnc{Figs/one_thing_per_cell_hilit1.pdf}{0.7}}

  \only<3|handout 0>{\fighnc{Figs/one_thing_per_cell_hilit2.pdf}{0.7}}

  \only<4|handout 0>{
    \fighnc{Figs/one_thing_per_cell.pdf}{0.7}

    \vspace*{-0.5\textheight}
    \hspace*{0.15\textwidth}
    \fighnc{Figs/one_thing_per_cell_rev.pdf}{0.7}
  }


  \note{
    Put just one value in each cell. The most common issues here are
    to include a note with a value. (Instead, put notes in a separate column.)
    Or you might want to include the units. (Instead, include the
    units in the column name, or even better include them in a
    separate data dictionary file, with metadata.)
  }
\end{frame}




\begin{frame}[c]{Make it a rectangle}

  \begin{columns}

    \column{0.5\textwidth}

    \figh{Figs/not_rectangle_1.pdf}{0.4}
    \bigskip

    \figw{Figs/not_rectangle_3.pdf}{1.0}

    \column{0.5\textwidth}

    \figh{Figs/not_rectangle_2.pdf}{0.4}
    \bigskip

    \figh{Figs/not_rectangle_4.pdf}{0.4}

  \end{columns}

  \note{
    The datasets I see tend to be organized in complex ways.
    Here are four examples.

    But when it comes to the layout of the data, what I want to see is
    just a rectangle, with subjects as rows, variables as columns, and
    a single header row.
  }
\end{frame}


\begin{frame}{Make it a rectangle}

    \fighnc{Figs/not_rectangle_1.pdf}{0.7}

    \only<2>{
      \vspace*{-0.45\textheight}
      \hspace*{0.30\textwidth}
      \fighnc{Figs/rectangle.pdf}{0.6}
    }

  \note{
    For example, this dataset has three variables presented as rows,
    each with its own row of IDs, and with blank lines between.

    A preferred layout would be to have a single column of IDs, and
    then another column for each attribute or measured variable.
  }
\end{frame}




\begin{frame}{Make it a rectangle}

    \fighnc{Figs/not_rectangle_2.pdf}{0.7}

    \only<2>{
      \vspace*{-0.55\textheight}
      \hspace*{0.20\textwidth}
      \fighnc{Figs/not_rectangle_2_corr1.pdf}{0.7}
    }

  \note{
    This example has data for a glucose tolerance test: a treatment
    was applied, and then glucose and insulin were measured from
    successive serum samples over time.

    The IDs are provided just in the rows for time=0. The assay date
    and the weight of the mouse are provided in those rows, too.

    One solution would be to fill in those ID, date, and weight
    measurements in each row. No empty cells!

    But this is a lot of duplicated information.
  }
\end{frame}




\begin{frame}{Make it a rectangle}

  \begin{columns}
    \column{0.35\textwidth}

    \figw{Figs/not_rectangle_2_corr2a.pdf}{1.0}

    \column{0.65\textwidth}

    \only<1|no handout>{
      \fighnc{Figs/not_rectangle_2_corr2b.pdf}{0.8}
    }
    \only<2>{
      \fighnc{Figs/not_rectangle_2_corr3b.pdf}{0.8}
    }

  \end{columns}

  \note{
    Another alternative is to split the data into two tables: one with
    assay date and mouse weight, and a second with the actual GTT
    data.

    We would also prefer those ``lo below curve'' notes pulled out as
    a separate column.
  }
\end{frame}


\begin{frame}{Make it a rectangle}

    \fighnc{Figs/not_rectangle_3.pdf}{0.55}

    \only<2->{
      \vspace*{-0.37\textheight}
      \hspace*{0.10\textwidth}
      \fighnc{Figs/not_rectangle_3_corr1.pdf}{0.55}
    }

    \only<3>{
      \vspace*{-0.37\textheight}
      \hspace*{0.20\textwidth}
      \fighnc{Figs/not_rectangle_3_corr2.pdf}{0.65}
    }

  \note{
    In this example, there are two header rows. One indicates week (4,
    6, or 8) for sets of three columns.

    Instead, we might include the week number in the column names, so
    that we just need a single header row.

    Alternatively, we could put each week in a separate row, and add a
    column with the week variable, so that each mouse's information is
    split across three rows.
  }
\end{frame}




\begin{frame}[c]{Make it a rectangle}

    \figh{Figs/not_rectangle_4.pdf}{0.8}


  \note{
    This spreadsheet is a single tab in a 500-tab Excel file, with one
    tab per animal. These data are similar to the GTT data. I'd
    probably split it into two files with days on diet and sex
    in one file and the experiment values in the other. And I would
    drop the calculations of mean, SD, and fold-change and just focus
    on a file with the raw measurements.
  }
\end{frame}







\begin{frame}[c]{}


  \LARGE
  \centering
  No calculations in the data file

  \note{
    And related to that last example: don't do calculations in your
    raw data file. Even if you find Excel useful for data analysis and
    visualization, it's best to do those things in a separate file,
    and keep your raw data pure and locked down, and opened just to
    add or correct data.

    Every time you open a data file, you introduce an opportunity for
    errors. So you only want to open them when you really need to.

    Have you ever opened an Excel file and started typing and nothing
    happens, and then you realize that you need to select a cell
    first? Well sometimes the stuff you were typing is entered in
    there in some sporadic location, for your data analyst to find
    later. I've seen some really weird bits of text typed into data
    files.
  }
\end{frame}




\begin{frame}[c]{Make a data dictionary}

  \figh{Figs/data_dict.pdf}{0.8}

  \note{
    Create a data dictionary, describing the variables in your
    dataset.

    I'd like to also have versions of the variable names useful in
    plots. It can also be useful to classify the variables or include
    other information about the measurement process, units, and allowed
    values/ranges.

    Metadata is data, so put it in a spreadsheet and make it a
    rectangle.
  }
\end{frame}




\begin{frame}[c]{No color/formatting as data}

\only<1>{
\hspace{0.1\textwidth}
  \fighnc{Figs/no_highlighting.pdf}{0.7} }

\only<2|handout 0>{
\hspace{0.1\textwidth}
\fighnc{Figs/no_highlighting_2.pdf}{0.7} }

  \note{
    Never use color or formatting as data. It can be tricky to extract
    such information.

    Rather, add an additional column, for example indicating
    questionable values.
  }
\end{frame}



\begin{frame}{Make backups}

  \bbi
\item Automatic
\item Multiple locations (including off site)
\item Consider formal version control
  \ei

  \note{
    Be sure to backup your data. Many of us use dropbox or google
    drive. The key things are that backups be automatic and in
    multiple locations including off site.

    If you need to insert an external drive for the backup to occur,
    it won't be done as frequently as it should. If it's not happening
    automatically, it won't get done as you'd like.

    And off site, so in case there's a fire or other accident in your
    lab or home, you don't lose everything.

    And consider a formal version control system like git and github.
    It's not ideal for large datasets, but it's good to be able to
    retain multiple versions of your data over time: to see what
    changes were made when, or to look at the state of the data on a
    particular date.
  }
\end{frame}



\begin{frame}[c]{Use data validation}

  \figh{Figs/data_validation.png}{0.65}

  \note{
    I don't have much experience with data entry, but if you're using
    excel to enter data, consider its data validation features, where
    you can indicate the allowed values in different columns, to help
    to catch data entry errors and to ensure consistency of variable
    categories.
  }
\end{frame}




\begin{frame}{Save as plain text}

  \bbi
\item Don't rely on a proprietary format
\item Save as a comma-delimited (CSV) or tab-delimited (TSV) file
  \bi
\item[] Or vertical-bar-delimited?
  \ei
  \ei

  \note{
    While Excel, OpenOffice, and Google Sheets formats are likely to
    be continued to be readable, I recommend saving data in
    non-proprietary plain-text formats, such as comma-delimited or
    tab-delimited text files, so that future users won't have to rely
    on the availability of particular software tools.
  }
\end{frame}




\begin{frame}{Summary}

  \begin{columns}

    \column{0.5\textwidth}

    \bbe
  \item Be consistent
  \item Write dates as {\tt YYYY-MM-DD}
  \item Choose good names for things
  \item No empty cells
  \item One thing per cell
  \item Make it a rectangle
    \ee

    \column{0.5\textwidth}

    \bbe
    \addtocounter{enumi}{6}
  \item Make a data dictionary
  \item No calculations in the data file
  \item No color/formatting as data
  \item Make backups
  \item Use data validation
  \item Save as plain text
    \ee

    \end{columns}


  \note{
    It's always good to have a summary.
  }

\end{frame}




\begin{frame}{Acknowledgements}


  \bbi
\item[] Kara Woo
\item[] Jenny Bryan
\item[] Hadley Wickham
\item[] All of my past scientific collaborators
  \ei

  \note{
    Thanks to Kara Woo, without whom this would never have become a
    proper paper.

    And thanks to Jenny Bryan and Hadley Wickham for encouraging us.

    And of course to my 20 years of collaborators and their creative
    uses of Excel.
  }

\end{frame}


\begin{frame}{Further reading}


{\small

  \bi
\item Broman KW, Woo KH (2018) Data organization in spreadsheets. Am Stat 72:2-10
 \href{https://doi.org/gdz6cm}{\tt doi.org/gdz6cm}
\item White EP et al. (2013) Nine simple ways to make it easier to
  (re)use your data. Ideas Ecol Evol 6:1-10
  \href{https://doi.org/10.4033/iee.2013.6b.6.f}{\tt doi.org/10.4033/iee.2013.6b.6.f}
\item Briney K (2015) Data management for researchers. Pelagic
  Publishing.
  \href{https://pelagicpublishing.com/products/data-management-for-researchers-briney}{ISBN: 9781784270117}
\item Ziemann et al (2016) Gene name errors are widespread in the
  scientific literature. Genome Biol 17:177
  \href{https://doi.org/10.1186/s13059-016-1044-7}{\tt doi.org/10.1186/s13059-016-1044-7}
\item Ellis SE, Leek JT (2018) How to share data for collaboration.
  Am Stat 72:53-57
  \href{https://doi.org/10.1080/00031305.2017.1375987}{\tt doi.org/10.1080/00031305.2017.1375987}
\item Wilson SL et al. (2021) Sharing biological data: why, when, and
  how. FEBS Letters 595:847-863
  \href{https://doi.org/10.1002/1873-3468.14067}{\tt doi.org/10.1002/1873-3468.14067}
  \ei

}

  \note{
    Here's our paper again, plus several other papers and
    a book which you may find interesting.
  }

\end{frame}


\begin{frame}[c]{}

\Large

Slides: \href{https://kbroman.org/Talk_DataOrg}{\tt kbroman.org/Talk\_DataOrg}
\hfill \includegraphics[height=7mm]{Figs/cc-zero.png}

\vspace{7mm}

\href{https://kbroman.org}{\tt \lolit kbroman.org}

\vspace{7mm}

\href{https://github.com/kbroman}{\tt \lolit github.com/kbroman}

\vspace{7mm}

\href{https://twitter.com/kwbroman}{\tt \lolit @kwbroman}


\note{
  Here's where you can find me and the slides.
}

\end{frame}




\end{document}
