\documentclass[aspectratio=169,12pt,t]{beamer}
\usepackage{graphicx}
\setbeameroption{hide notes}
\setbeamertemplate{note page}[plain]
\usepackage{listings}
\usepackage{eepic}

\input{header.tex}

%%%%%%%%%%%%%%%%%%%%%%%%%%%%%%%%%%%%%%%%%%%%%%%%%%%%%%%%%%%%%%%%%%%%%%
% end of header
%%%%%%%%%%%%%%%%%%%%%%%%%%%%%%%%%%%%%%%%%%%%%%%%%%%%%%%%%%%%%%%%%%%%%%

% title info
\title{data organization in spreadsheets}
\subtitle{}
\author{\href{https://kbroman.org}{Karl Broman}}
\institute{Biostatistics \& Medical Informatics, UW{\textendash}Madison}
\date{\href{https://kbroman.org}{\tt \scriptsize \color{foreground} kbroman.org}
\\[-4pt]
\href{https://github.com/kbroman}{\tt \scriptsize \color{foreground} github.com/kbroman}
\\[-4pt]
\href{https://twitter.com/kwbroman}{\tt \scriptsize \color{foreground} @kwbroman}
\\[-4pt]
{\scriptsize Slides: \href{https://kbroman.org/Talk_DataOrg}{\tt kbroman.org/Talk\_DataOrg}}
}


\begin{document}

% title slide
{
\setbeamertemplate{footline}{} % no page number here
\frame{
  \titlepage

  \vfill \hfill \includegraphics[height=6mm]{Figs/cc-zero.png} \vspace*{-5mm}

\note{
  These are slides for a talk for the OSGA Webinar Series
  on 24 Sept 2021, based on
  my paper of the same title with Kara Woo,
  {\tt doi.org/gdz6cm}.
}

} }



\begin{frame}[c]{}


  \figh{Figs/spreadsheet_ugly.pdf}{0.8}


  \note{
    Spreadsheets are super useful for storing and organizing datasets.
    But how should you arrange the data in a spreadsheet?

    Often, data are arranged as they might appear in a table in a
    paper. This may be pleasing to view, but it can be make
    down-stream analysis difficult.

    Data analysts often spend a lot of time rearranging data prior to
    analysis. Here, I'll take about how to organize data in
    spreadsheets, for ease of analysis.
  }

\end{frame}





\begin{frame}[c]{}


  \figh{Figs/dataorg_paper.png}{0.7}

  \vspace{10mm}
  \hfill \href{https://doi.org/gdz6cm}{\tt \lolit doi.org/gdz6cm}
  \vspace{-10mm}

  \note{
    A particular collaborator's data (not that shown on the previous
    slide) led me to write a website on data organization, which then
    led to this paper.
  }

\end{frame}





\begin{frame}[c]{}

  \centering
  \Large

  Be consistent

  \note{
    First, be consistent.
  }

\end{frame}


{
% invert colors on this slide
\setbeamercolor{normal text}{bg=foreground, fg=background}
\color{background}

\begin{frame}<handout:0>[c]{}

  \addtocounter{framenumber}{-1}

  \centering
  \fontsize{66pt}{66}\selectfont

  Be consistent

\end{frame}
}


\begin{frame}[c]{Consistent categories}

  % spreadsheet with sex categories, missing values, and subject IDs
  % highlight the columns/values

  \note{

  }

\end{frame}



\begin{frame}[c]{Consistent missing values}

  % emphasize: no -999 or 999
  % use "-" or "NA" or "N/A"
  % "NA" can be problemmatic because "north america"

  \note{

  }

\end{frame}



\begin{frame}[c]{Consistent subject IDs}

  % 394, mouse-394, mouse394F, Mouse0394

  \note{

  }

\end{frame}




\begin{frame}{Consistent column names}

\includegraphics[height=0.8\textheight]{Figs/spreadsheet_colnames1.pdf}

\vspace*{-0.6\textheight}
\hspace*{0.1\textwidth}
\includegraphics[height=0.8\textheight]{Figs/spreadsheet_colnames2.pdf}

  \note{

  }

\end{frame}



\begin{frame}{Consistent layout}


\includegraphics[height=0.8\textheight]{Figs/spreadsheet_colnames1.pdf}

\vspace*{-0.6\textheight}
\hspace*{0.1\textwidth}
\includegraphics[height=0.8\textheight]{Figs/spreadsheet_colnames2.pdf}

\vspace*{-0.6\textheight}
\hspace*{0.2\textwidth}
\includegraphics[height=0.8\textheight]{Figs/spreadsheet_colnames3.pdf}

  \note{

  }

\end{frame}



\begin{frame}[c]{Consistent notes}

  % spreadsheet with various notes


  \note{

  }

\end{frame}



\begin{frame}[c]{Consistent date format}

  % spreadsheet with crazy dates


  \note{

  }

\end{frame}



\begin{frame}[c]{}

\vspace{24pt}

\figh{Figs/iso_8601.png}{0.8}

\vfill

\hfill {\tt \footnotesize \lolit \href{http://xkcd.com/1179/}{xkcd.com/1179}}



  \note{
    Go with the xkcd format for writing dates, for ease of sorting.
  }

\end{frame}





\begin{frame}[c]{}

  \centering

{
% fbox to put border around the images
\setlength{\fboxsep}{0pt}
\setlength{\fboxrule}{1pt}

\begin{minipage}{0.7\textwidth}
  % using minipage so fbox is tight around figure
  % (otherwise it spans width of slide even if fig doesn't)

\fbox{\figw{Figs/oct4_tweet_1.png}{1.0}}

\fbox{\figw{Figs/oct4_tweet_2.png}{1.0}}
\end{minipage}
}

  \note{

  }

\end{frame}






\begin{frame}[c]{Consistent file names}

  % give examples
  % general principles:
  % - no spaces or special characters
  % - short but descriptive
  % - consistent scheme
  % - take advantage of computer sorting


  \note{

  }

\end{frame}



\begin{frame}[c]{No ``{\hilit final}'' in file names}

\vspace*{3mm}

\centering

% comic from http://www.phdcomics.com/comics/archive.php?comicid=1531
\figh{Figs/phd101212s.png}{0.8}

\note{
  Never include ``final'' in a file name.
}

\end{frame}




\begin{frame}[c]{Choose good names for things}

  % table in the paper?
  % rules: no spaces or special characters; short but descriptive


  \note{

  }

\end{frame}




\begin{frame}[c]{Summary}


  % list all of the key principles, in two columns

  \note{

  }

\end{frame}






\begin{frame}[c]{}

\Large

Slides: \href{https://kbroman.org/Talk_DataOrg}{\tt kbroman.org/Talk\_DataOrg}
\hfill \includegraphics[height=7mm]{Figs/cc-zero.png}

\vspace{7mm}

\href{https://kbroman.org}{\tt \lolit kbroman.org}

\vspace{7mm}

\href{https://github.com/kbroman}{\tt \lolit github.com/kbroman}

\vspace{7mm}

\href{https://twitter.com/kwbroman}{\tt \lolit @kwbroman}


\note{
  Here's where you can find me and the slides.
}

\end{frame}




\end{document}
